\documentclass[12pt]{article}

\usepackage{booktabs}
\usepackage[table,x11names,dvipsnames,table]{xcolor}
\usepackage[scientific-notation=true]{siunitx} 
\usepackage{graphicx} 
\usepackage{natbib} 
\usepackage{amsmath} 
\usepackage[spanish,activeacute]{babel}
\usepackage{float}
\usepackage[table]{xcolor}
\usepackage{anysize}
\usepackage{caption}
\usepackage{subcaption}
\usepackage{pdfpages}
\marginsize{1in}{1in}{1in}{1in} 
\usepackage{enumerate}
\setlength\parindent{30pt}
\usepackage[utf8]{inputenc}

\newcommand{\HRule}{\rule{\linewidth}{0.5mm}}
%\setlength\parindent{0pt} 

\renewcommand{\labelenumi}{\alph{enumi}.}

\newcolumntype{a}{>{\columncolor[gray]{0.9}}c}



\begin{document}
%-----------------------------------------------------------

\begin{center}


\includegraphics[width=0.3\textwidth]{unison-logo.png}~\\[1cm]

\textsc{\LARGE Universidad de Sonora}\\[0.1cm]
\textsc{Divisi\'on de Ciencias Exactas y Naturales}\\[0.1cm]
\textsc{Departamento de F\'isica}\\[1.5cm]

\HRule \\[0.4cm]
\textsc{Física computacional I}\\[0.1cm]
\textsc{Actividad 1- Preguntas sobre \LaTeX}
\HRule \\[1.5cm]





\textsc{Jessica Isamar Uriarte García\\[1.0cm]}

\textsc{Docente:\\Carlos Lizarraga-Celaya\\[0.1cm]}
\vfill
\textsc{\today \\[0.1cm]}
\end{center}
\newpage
%-----------------------------------------------------------

\begin{abstract}
\noindent En esta sección responderé una serie de preguntas acerca del tipógrafó electrónico \LaTeX.
\end{abstract}

\section*{Preguntas:}
\begin{enumerate}
\item ¿Cual es tu primera impresión de uso de LaTeX?\\
Los trabajos se ven muy profesionales, mis ideas se ven más organizadas a comparación con mis trabajos en Office Word. \\

\item ¿Qué aspectos te gustaron más?

La variedad de documentos que se pueden crear; beamers, bibliográficos, artículos, etc. \\

\item ¿Qué no pudiste hacer en LaTeX?

Encontrar un paquete que facilitara agregar mi bibliografía estilo APA. \\

\item En tu experiencia, comparado con otros editores, ¿cómo se compara LaTeX? 

No es un editor gráfico, se maneja con commandos. \\

\item ¿Qué es lo que mas te llamó la atención en el desarrollo de esta actividad?

La estructura de la atmósfera. \\

\item ¿Qué cambiarías en esta actividad?

Juntaría las preguntas sobre latex y el resumen de la estructura de la atmósfera en el mismo PDF. \\

\item ¿Que consideras que falta en esta actividad? 

No considero que le falte algo a esta actividad, es muy completa.\\

\item ¿Algún comentario adicional que desees compartir? \\
No.

\end{enumerate}
\end{document}
