%%%%%%%% PAQUETES NECESARIOS
\documentclass[12pt]{article}
\usepackage{booktabs}
\usepackage[table,x11names,dvipsnames,table]{xcolor}
\usepackage[scientific-notation=true]{siunitx} 
\usepackage{graphicx} 
\usepackage{natbib} 
\usepackage{amsmath} 
\usepackage[spanish,activeacute]{babel}
\usepackage{float}
\usepackage[table]{xcolor}
\usepackage{anysize}
\usepackage{caption}
\usepackage{subcaption}
\usepackage{amsmath,enumerate}
\usepackage{pdfpages}
\marginsize{1in}{1in}{1in}{1in} 
\usepackage{enumerate}
\setlength\parindent{30pt}
\usepackage[utf8]{inputenc}

\usepackage[margin=1.5in]{geometry}
\usepackage{amsthm}
\usepackage{amsfonts}
\usepackage{amssymb}
\usepackage{tikz}
\usepackage[colorinlistoftodos, color=orange!50]{todonotes}
\usepackage{hyperref}
\usepackage{fancybox}
\usepackage{epsfig}
\usepackage{soul}
\usepackage[framemethod=tikz]{mdframed}
\usepackage[shortlabels]{enumitem}
\usepackage[version=4]{mhchem}
\usepackage{multicol}
\usepackage{booktabs}
\usepackage{epigraph}
\usepackage{wrapfig}


\newcommand{\HRule}{\rule{\linewidth}{0.4mm}}
%\setlength\parindent{0pt} 
\renewcommand{\labelenumi}{\alph{enumi}.}
\newcolumntype{a}{>{\columncolor[gray]{0.9}}c}

%%%%%%%% TITULO 
\title{FC-Actividad5-Emacs}






\begin{document}

%#################################################################

\begin{center}


\includegraphics[width=0.3\textwidth]{unison-logo.png}~\\[0.8cm]

\textsc{\LARGE Universidad de Sonora}\\[0.1cm]
\textsc{Divisi\'on de Ciencias Exactas y Naturales}\\[0.1cm]
\textsc{Departamento de F\'isica \\[0.5cm] \textbf{F\'isica Computacional I}}\\%[0.5cm]

\end{center}
\noindent
Actividad 5 \dotfill \textbf{\underline{Preparando datos con Emacs}}\\
Estudiante \dotfill Jessica Isamar Uriarte Garc\'ia \\
Docente \dotfill Carlos Liz\'arraga Celaya \\
Fecha \dotfill \today \\
\rule{\linewidth}{0.5pt} \\[6pt] 
\abstractname{\\ \emph{ \scriptsize{  \baselinestretch En esta actividad haremos uso intensivo del Editor Emacs para preparar los datos de sondeos de la Atm\'osfera descargados en la actividad anterior,  para su an\'alisis posterior con Panda.\\} }}
\renewcommand{\baselinestretch}{0.6}
\rule{\linewidth}{2pt}  
\small\tableofcontents

\newpage
%#################################################################
%\begin{figure}[H]
%\includegraphics[width=0.9\textwidth]{img1.png}
%\centering
%\caption{\emph{\scriptsize{...}}}
%\end{figure}
%#################################################################

\section{Introducci\'on}
\noindent 

%#################################################################

\section{Índice CAPE \emph{(Convective available potential energy)}}
\noindent

\subsection*{Diccionario de Spain Severe Weather sobre el Índice LI (Lifted Index):} 

"Medida común de la inestabilidad atmosférica. Su valor se obtiene por el cálculo de la temperatura que el aire cerca del suelo tendría si fuera elevado a algún nivel más alto (alrededor de 18,000 pies, normalmente) y comparando esta temperatura con la actual a ese nivel. Valores negativos indican inestabilidad – cuanto más negativo, más inestable es el aire, y más fuertes podrían ser los chorros ascendentes con cualquier desarrollo de tormentas. No obstante no hay números mágico o valores LI umbral debajo de los cuales el tiempo severo se haga inminente."
%#################################################################

\section{Agua precipitable}
\noindent



%#################################################################

\section{Conclusi\'on}
\noindent

%#################################################################

\pagebreak
\section{Bibliograf\'ia}
\noindent

\begin{enumerate} [\hspace{16pt} 1.]
		\item Steve Parker (2017). Shell Script Tutorial (shellscript.sh)

		\item Wikipedia. (2018). Shell Script. 2018, de Wikipedia Sitio web: \url{https://en.wikipedia.org/wiki/Shell_script}
        
        \item HowtoForge. (2018). 1 The GREP command- an overview. 2018, de LINUX Tutorials Sitio web: \url{https://www.howtoforge.com/tutorial/linux-grep-command/}
        
        \item Wikipedia. (2018). Cat (Unix). 2018, de Wikipedia Sitio web: \url{https://en.wikipedia.org/wiki/Cat_(Unix)}
        
        \item Wikipedia. (2018). less (Unix). 2018, de Wikipedia Sitio web: \url{https://en.wikipedia.org/wiki/Less_(Unix)}
        
        \item Wikipedia. (2018). chmod. 2018, de Wikipedia Sitio web: \url{https://es.wikipedia.org/wiki/Chmod}
        
\end{enumerate}

%#################################################################
\section{Ap\'endice}
\begin{itemize}
\item  ¿Cómo se te hizo esta actividad? ¿Compleja, Difícil, Sencilla?

No estuvo dif\'icil pero tuve dudas y no las aclar\'e a tiempo.
\item ¿Qué te llamó más la atención?

Con las gr\'aficas terminadas podr\'e ver los meses donde m\'as tormentas hubo en Tucson.
\item ¿Qué parte fue la que menos te interesó hacer?

Supe borrar informaci\'on que no se ocupa en emacs con query-replace pero no supe como darle la forma de base de datos a lo que me sobr\'o.
\item ¿Cómo mejorarías esta actividad? ¿Qué le faltó? ¿Qué sobró?

La actividad estuvo bien, a mi me falt\'o tiempo para terminarlo.
\item ¿Hasta este punto, que te parece el uso de Jupyter para programar en Python?  

Me parece m\'as r\'apido y f\'acil de compilar que otras apps como Spyder o CodeBlocks. 
\end{itemize}
%#################################################################
\end{document}